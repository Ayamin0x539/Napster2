\documentclass[a4paper,12pt]{article}
\linespread{1.2}

\begin{document}

\title{CSE4701: Final Project Diary}
\date{}
\author{Jonathan Roemer}
\maketitle

\section{Tuesday, 1 December 2015}
\subsection{System Configuration and Tool Installation}
Though our team was coordinating over Telegram, members of the team had reached
different stages of the system configuration process. We had previously decided
on the following tools:
\begin{enumerate}
	\item Django as our web framework
	\item Python 3 as our base
	\item Our modified Chinook database from Phase 1
	\item MySQL Workbench to interact with the Chinook database
	\item Ubuntu as our base, as both 12.04 and 14.04 support the packages we 
		needed.
\end{enumerate}

However, there was no clear documentation on how to replicate our installation
process. Therefore, I went through a fresh installation with the team present.
We were able to present a concise and easy to follow set of instructions for
not only running our web application, but also configuring a development
environment for the application.

This included both writing out the installation steps and creating a
Chinook_startingstate.sql file. This file includes our custom version of Chinook
in a format that can be easily imported and integrated with django.

\subsection{Content Planning}
Using our README as an inprompteu to-do list, we went through the assignment
requirements and delegated reponsibility for different views and functionality.
I begun work on the search functionality, primarily by familiarizing myself with
django. I also applied some basic CSS to the application, though, because of how
django sources directories, it was more difficult than I had anticipated to even
add a global CSS file and background image. I may simply stick with a small
section of CSS embedded directly into the base.html file.

\section{Wednesday, 2 December 2015}

As noted in Tuesday's entry, having a global CSS and image resource file proved
to be surprisingly difficult because of how django sources the present working
directory. I was able to solve this by moving all of our static resources and
templates into the napster2 application directory. That way, I did not need to
navigate out of the application directory to source resources. The only other
changes required were adding \{\% load staticfiles\%\} to the base.html template
and correctly using \{\% static 'theme/css/base.css' \%\} to source the base.css
file.

\end{document}
